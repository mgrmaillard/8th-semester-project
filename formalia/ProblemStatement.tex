\documentclass[main.tex]{subfiles}

\begin{document}

\setlength{\abovedisplayskip}{-15pt}
\setlength{\belowdisplayskip}{0pt}
\setlength{\abovedisplayshortskip}{0pt}
\setlength{\belowdisplayshortskip}{0pt}
\allowdisplaybreaks[2]

\chapter{Introduction}
The most well known option pricing model within financial world is the Black-Scholes model, where it is assumed that the price process follows a geometric brownian motion.

\begin{align*}
dS(t) = \alpha S(t) dt + \sigma S(t) dB(t)
\end{align*}

where $\alpha$ is a constant drift, $\sigma$ is a constant volatility and $B(t)$ is a Brownian motion. 
There are several problems with this specific type of dynamic structure for the price process. One of the well known problems is the assumption of the constant volatility, which was showed to not hold, as volatility of price process varies through out time.

\subsection*{Research Question}


\end{document}