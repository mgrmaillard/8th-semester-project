\RequirePackage{etex}

\usepackage[T1]{fontenc}
% Hjælper med orddeling ved æ, ø og å. Sætter fontene til at være ps-fonte, i stedet for bmp	
\usepackage[utf8]{inputenc} 

%	¤¤ Blank pages ¤¤ %
\usepackage{afterpage} % Blank page by \afterpage{\blankpage}

\newcommand\blankpage{%
    \null
    \thispagestyle{empty}%
    \addtocounter{page}{-1}%
    \newpage}


\usepackage{listings}
\usepackage{lastpage}


\usepackage[footnote,draft,danish,silent,nomargin]{fixme}
%	¤¤ Usepackages ¤¤ %
%\usepackage[danish]{babel}
% Dansk sporg, f.eks. tabel, figur og kapitel
\usepackage{booktabs}
\usepackage{amsmath}
\usepackage{array, booktabs}
\usepackage{framed}

%Til horisontale tabeller
\usepackage{lscape}
\usepackage{rotating}

\usepackage{arydshln} %Dash line 

\usepackage[framemethod=tikz]{mdframed}
\pretolerance=2500 										
% Gør det muligt at justere afstanden mellem ord (højt tal, mindre orddeling og mere space mellem ord)
\usepackage{amssymb, amsthm, mathtools, bm}
\usepackage{pdfpages}										% Gør det muligt at inkludere pdf-dokumenter med kommandoen \includepdf[pages=]{fil.pdf}	
\usepackage{varioref}
\usepackage[hidelinks]{hyperref}			 	% Giver mulighed for at ens referencer bliver til klikbare hyperlinks. .. [colorlinks]{..}

\usepackage{nameref, cleveref}


%\usepackage[demo]{graphicx}
\usepackage{subfig}

%\usepackage{subcaption}

\usepackage{etoolbox}
% forskellige pakker til brug når I skriver matematik
\usepackage{graphicx}
% % Pakke til ekstern jpeg/png billeder

%% COMMANDS %%
\newcommand{\overbar}[1]{\mkern 1.5mu\overline{\mkern-1.5mu#1\mkern-1.5mu}\mkern 1.5mu}

\newcommand{\msout}[1]{\text{\sout{\ensuremath{#1}}}}
\newcommand{\Var}[1]{\,\mathrm{Var}\left[\,#1\,\right]}
\newcommand{\Cov}[1]{\,\mathrm{Cov}\left[\,#1\,\right]}
\newcommand{\Corr}[1]{\,\mathrm{Corr}\left(\,#1\,\right)}
\newcommand{\diff}[2]{\dfrac{d #1}{d #2}}
\newcommand{\pd}[2]{\dfrac{\partial #1}{\partial #2}}


\newcommand{\of}[1]{\left(#1\right)}
\newcommand{\off}[1]{\left[#1\right]}%laver en command til automatisk store parateser
\newcommand{\Hy}{\mathcal{H}} %Laver fancy H til hypotese
\newcommand{\m}{\cdot}

\newcommand{\VaR}{\mathrm{VaR}}

\newcommand{\Nd}[1]{\mathcal{N}\!\left(#1\right)}
\newcommand{\Nddim}[2][n]{\mathcal{N}_#1\!\left(#2\right)}
\newcommand{\wn}[1]{\mathrm{wn}\!\left(#1\right)}

\renewcommand{\hat}[1]{\widehat{#1}} %stor hat
\renewcommand{\bar}[1]{\overbar{#1}} %stor bar
\renewcommand{\tilde}[1]{\widetilde{#1}} %stor tilde

\newcommand{\Norm}[1]{\lVert#1\rVert}
\newcommand{\E}[1]{\,\mathbb{E}\left[\,#1\,\right]} 
%\newcommand{\Prob}[1]{\,\mathrm{P}\left(#1\right)} 
\newcommand{\R}{\mathbb{R}}
\newcommand{\F}{\mathcal{F}}
\newcommand{\B}{\mathcal{B}}
\newcommand{\T}{\mathcal{T}}
\newcommand{\D}{\mathcal{D}}
\newcommand{\Lik}{\mathcal{L}}
\newcommand{\N}{\mathbb{N}}
\newcommand{\I}{\mathrm{I}}
\newcommand{\U}{\mathrm{U}}
\newcommand{\iid}{\mathrm{i.i.d.}}
\newcommand{\SSR}{\,\mathrm{SSR}}
\newcommand{\MA}{\mathrm{MA}}
\newcommand{\AR}{\mathrm{AR}}
\newcommand{\ARMA}{\mathrm{ARMA}}
\newcommand{\ARIMA}{\mathrm{ARIMA}}
\newcommand{\ARMAX}{\mathrm{ARMAX}}
\newcommand{\ARCH}{\mathrm{ARCH}}
\newcommand{\GARCH}{\mathrm{GARCH}}
\newcommand{\EGARCH}{\mathrm{EGARCH}}
\newcommand{\FIGARCH}{\mathrm{FIGARCH}}
\newcommand{\IGARCH}{\mathrm{IGARCH}}
\newcommand{\norm}{\mathrm{norm}}
\newcommand{\std}{\mathrm{std}}
\newcommand{\sstd}{\mathrm{sstd}}
\newcommand{\snorm}{\mathrm{snorm}}
\newcommand{\MSE}{\mathrm{MSE}}
\newcommand{\VIF}{\operatorname{VIF}}
\newcommand{\se}[1]{\operatorname{se}\!\left(#1\right)}
\newcommand{\sd}[1]{\operatorname{sd}\!\left(#1\right)}
\newcommand{\tr}{\operatorname{tr}}
\newcommand{\dash}{\operatorname{-}}
\newcommand{\conv}[1]{\overset{#1}{\longrightarrow}}
%\newcommand{\logicals}[1]{\textcolor{blue!60!black}{\ttfamily{#1}}}
%\newcommand{\strings}[1]{\textcolor{lightblue}{\ttfamily{#1}}}
\newcommand{\rfont}[1]{\textcolor{blue}{\ttfamily{#1}}}
\newcommand{\rfontblack}[1]{\textcolor{black}{\ttfamily{#1}}}


% Independence sign
\newcommand\independent{\protect\mathpalette{\protect\independenT}{\perp}}
\def\independenT#1#2{\mathrel{\rlap{$#1#2$}\mkern2mu{#1#2}}}

\DeclareMathOperator*{\argmax}{arg\,max}
\DeclareMathOperator*{\argmin}{arg\,min}

% Floor brackets til greatest integer function
\DeclarePairedDelimiter{\floor}{\lfloor}{\rfloor}

%%-------------------------------------------------%%


\usepackage{tabularx,ragged2e}
\usepackage{lscape}
\newcolumntype{C}{>{\Centering\arraybackslash}X}
\newcolumntype{L}{>{\RaggedRight\arraybackslash}X}
\newcolumntype{b}{>{\Centering\hsize=2.3\hsize}X}
\newcolumntype{s}{>{\Centering\hsize=.3\hsize}X}
\newcolumntype{m}{>{\Centering\hsize=.9\hsize}X}


\usepackage[normalem]{ulem}
\usepackage{pdfpages}


\definecolor{lightblue}{RGB}{100,100,255}
\definecolor{string}{RGB}{0,0,255}

%	¤¤ Compiling settings ¤¤ %

\graphicspath{{Images/}{../../../../Images/}{../../Images/}} % Brug denne til at finde directory
									
									
% ¤¤ Marginer ¤¤ %
\setlrmarginsandblock{3.5cm}{2.5cm}{*}	
% \setlrmarginsandblock{Indbinding}{Kant}{Ratio}
\setulmarginsandblock{2.5cm}{3.0cm}{*}	
% \setulmarginsandblock{Top}{Bund}{Ratio}
\checkandfixthelayout 									
% Laver forskellige beregninger og sætter de almindelige længder op til brug ikke memoir pakker

%	¤¤ Afsnitsformatering ¤¤ %
\setlength{\parindent}{0mm}           	
% Størrelse af indryk
\setlength{\parskip}{3mm}          			
% Afstand mellem afsnit ved brug af double Enter
\linespread{1,1}											
% Linie afstand

% ¤¤ Sidehoved ¤¤ %
%\pagestyle{plain}

% ¤¤ Indholdsfortegnelse ¤¤ %
\setsecnumdepth{subsection}		 			
% Dybden af nummerede overkrifter (part/chapter/section/subsection)
\maxsecnumdepth{subsection}					
% Ændring af dokumentklassens grænse for nummereringsdybde
\settocdepth{subsection} 							
% Dybden af indholdsfomrtegnelsen

\definecolor{gray}{gray}{0.80}					
% Definerer farven grå

%\definecolor{greentitle}{RGB}{194,193,204}
\definecolor{AAUBlue}{RGB}{33,26,82}
% Definerer farven AAUBlue 

% ¤¤ Kapiteludssende ¤¤ %
\definecolor{numbercolor}{gray}{0.7}
\newif\ifchapternonum

\makeatletter
\renewcommand*\env@matrix[1][*\c@MaxMatrixCols c]{%
  \hskip -\arraycolsep
  \let\@ifnextchar\new@ifnextchar
  \array{#1}}
\makeatother


\usepackage{titlesec}
\usepackage{lipsum}

\newcommand{\hsp}{\hspace{10pt}}

\makeatletter
\def\thickhrulefill{\leavevmode \leaders \hrule height 0.65ex \hfill \kern \z@}

\def\@makechapterhead#1{%
  %\vspace*{50\p@}%
  \vspace*{-30\p@}%
  {\parindent \z@ \centering \reset@font
        \thickhrulefill\quad
        \scshape\thechapter
        \quad \thickhrulefill
        \par\nobreak
        \vspace*{10\p@}%
        \interlinepenalty\@M
        \hrule
        %\vspace*{10\p@}%
        \huge \bfseries #1\par\nobreak
        %\par
        \vspace*{10\p@}%
        \hrule
    \vskip 8\p@
    %\vskip 100\p@
  }}
  
  \def\@makeschapterhead#1{%
  %\vspace*{50\p@}%
  \vspace*{-30\p@}%
  {\parindent \z@ \centering \reset@font
        \scshape
        \par\nobreak
        \vspace*{10\p@}%
        \interlinepenalty\@M
        \hrule
        %\vspace*{10\p@}%
        \huge \bfseries #1\par\nobreak
        %\par
        \vspace*{10\p@}%
        \hrule
    \vskip 8\p@
    %\vskip 100\p@
  }}

% \makeatletter
% \let\ps@plain\ps@NoHeader
% \makeatother

% \def\my@foot{\hbox to \textwidth{\rlap{\rule[2ex]{\textwidth}{0.4pt}}\thepage}}
% \def\ps@mine{\ps@empty% clear all current headings and footings
%     \let\@oddfoot\my@foot\let\@evenfoot\my@foot
%     \def\@oddhead{Contents\hfill\includegraphics[height=22pt]{example-image-a}}
%     \def\@evenhead{\includegraphics[height=22pt]{example-image-a}\hfill Contents}
% }
% \def\ps@plain{% this seems to be the "first" page for report.cls
%     \ps@empty\let\@oddfoot\my@foot\let\@evenfoot\my@foot
% }

% \usepackage{tocloft}
% \usepackage[english]{babel}
% \addto\captionsenglish{% 
%   \renewcommand{\contentsname}%
%     {\scshape \centering Table Of Contents}
% }

\renewcommand{\printtoctitle}{\centering\Huge\bfseries}
\renewcommand{\contentsname}{\vspace*{-30\p@} \scshape \hrule \vspace*{10\p@} Table of Contents \vspace*{10\p@} \hrule}

\usepackage{titleps}

\newpagestyle{projekt}{
  \setheadrule{.4pt}% Header rule
  \setfootrule{.4pt}% Footer rule
  \sethead[]% odd-left
          []% odd-center
          []% odd-right
          {\scshape Chapter \thechapter: \chaptertitle}% even-left
          {}% even-center
          {}% even-right
  \setfoot[]% odd-left
          [\thepage]% odd-center
          []% odd-right
          {}% even-left
          {\thepage}% even-center
          {}% even-right
}

\newpagestyle{NoHeader}{
  %\setheadrule{.4pt}% Header rule
  \setfootrule{.4pt}% Footer rule
  \sethead[]% odd-left
          []% odd-center
          []% odd-right
          {}% even-left
          {}% even-center
          {}% even-right
  \setfoot[]% odd-left
          [\thepage]% odd-center
          []% odd-right
          {}% even-left
          {\thepage}% even-center
          {}% even-right
}
  
% \usepackage{fancyhdr}
% \pagestyle{fancy}
% \renewcommand{\sectionmark}[1]{\markboth{#1}{}} % set the \leftmark
% \fancyhf{}
% \fancyhead{}
% \fancyhead[RO, LE]{\leftmark} % 1. sectionname
% \fancyfoot[C]{\thepage}
% \fancypagestyle{plain}{%
% \fancyhead{}%
% \renewcommand{\headrulewidth}{0pt}
% }  

% \usepackage{fancyhdr}% http://ctan.org/pkg/fancyhdr
% \renewcommand{\sectionmark}[1]{\markboth{#1}{}}
% \pagestyle{fancy}% Change page style to fancy
% \fancyhf{}% Clear header/footer
% \fancyhead[L]{\scshape \leftmark} 
% \fancyhead[R]{}
% \fancyfoot[L]{Footer}
% \fancyfoot[R]{\thepage}
% \renewcommand{\headrulewidth}{1pt}
% \renewcommand{\footrulewidth}{0.5pt}
  
 %\def\chaptermark##1{\markboth{% \ifnum \value{secnumdepth} < -1 \if@mainmatter \chaptername\ \thechapter\ --- % \fi \fi ##1}{}} 
% \nouppercaseheads
% \makepagestyle{projekt}
% \makeoddhead{projekt}{Statistical Modelling and Analysis}{}{Group 5.242}
% \makeoddfoot{projekt}{}{\thepage}{}


%\titleformat{\chapter}[hang]{\fontsize{26pt}{1cm}\bfseries}{\thechapter\hsp\textcolor{AAUBlue}{|}\hsp}{0pt}{}
%\titlespacing*{\chapter}{0pt}{-30pt}{5pt}

% ¤¤ Section ¤¤ %
\setsecheadstyle{\LARGE\bfseries}
\titlespacing*{\section}{0pt}{12pt plus 4pt minus 2pt}{0pt plus 2pt minus 2pt}
\setsubsecheadstyle{\Large\bfseries}
\titlespacing*{\subsection}{0pt}{12pt plus 4pt minus 2pt}{-4pt plus 2pt minus 2pt}
\setsubsubsecheadstyle{\large\bfseries}
\titlespacing*{\subsubsection}{0pt}{12pt plus 4pt minus 2pt}{-4pt plus 2pt minus 2pt}
\setparaheadstyle{\large\bfseries}
\setsubparaheadstyle{\large\bfseries}

% ¤¤ Indholdsfortegnelse ¤¤ %
%\addto\captionsdanish{
%	\renewcommand\contentsname{Indholdsfortegnelse}}

% ¤¤ Matematik ¤¤ %
\renewcommand{\qedsymbol}{$\blacksquare$} % sort QED tegn

\newtheoremstyle{theorem}% name of the style to be used
  {}% measure of space to leave above the theorem. E.g.: 3pt
  {}% measure of space to leave below the theorem. E.g.: 3pt
  {\em}% name of font to use in the body of the theorem
  {}% measure of space to indent
  {\bfseries}% name of head font
  {}% punctuation between head and body
  {\newline}% space after theorem head; " " = normal interword space
  {}% Manually specify head
  
\theoremstyle{theorem}
\newtheorem{theorem}{Theorem}[chapter] % Start sætning \begin{theorem}
\newtheorem{lemma}[theorem]{Lemma} % Start lemma \begin{lemma}
\newtheorem{corollary}[theorem]{Corollary} %Star korollar \begin{korollar}
\newtheorem{prop}[theorem]{Proposition}


\newtheoremstyle{proof}% name of the style to be used
  {-10pt}% measure of space to leave above the theorem. E.g.: 3pt
  {-5pt}% measure of space to leave below the theorem. E.g.: 3pt
  {}% name of font to use in the body of the theorem
  {}% measure of space to indent
  {\em}% name of head font
  {}% punctuation between head and body
  {}% space after theorem head; " " = normal interword space
  {}% Manually specify head
  
\theoremstyle{proof}
 
\renewenvironment{proof}{{\bfseries \itshape Proof.}}{\qed}

\newtheoremstyle{mydef}% name of the style to be used
  {}% measure of space to leave above the theorem. E.g.: 3pt
  {}% measure of space to leave below the theorem. E.g.: 3pt
  {}% name of font to use in the body of the theorem
  {}% measure of space to indent
  {\bfseries}% name of head font
  {}% punctuation between head and body
  {\newline}% space after theorem head; " " = normal interword space
  {}% Manually specify head

\definecolor{shadecolor}{gray}{0.925}
\theoremstyle{mydef}
\newtheorem{mydef}{Definition}[chapter] % Start definition \begin{mydef}
\surroundwithmdframed[backgroundcolor=shadecolor,linewidth=0.8pt, nobreak=true]{mydef}


%\AtEndEnvironment{theorem}{\null\hfill\rule{45pt}{1pt}}%
%\AtEndEnvironment{proposition}{\null\hfill\rule{45pt}{1pt}}%
%\AtEndEnvironment{corollary}{\null\hfill\rule{45pt}{1pt}}%
%\AtEndEnvironment{lemma}{\null\hfill\rule{45pt}{1pt}}%



%{spaceabove}% measure of space to leave above the theorem. E.g.: 3pt
%{spacebelow}% measure of space to leave below the theorem. E.g.: 3pt
%{bodyfont}% name of font to use in the body of the theorem
%{indent}% measure of space to indent
%{headfont}% name of head font
%{headpunctuation}% punctuation between head and body
%{\n}% space after theorem head; " " = normal interword space
%{headspec}% Manually specify head

%\newenvironment{foo}[2]{\textbf{#1} \newline}{}

\usepackage{eurosym}
\usepackage{latexsym}

% ¤¤ Eksempel mm. ¤¤ %
\newtheorem{example}{Example}[chapter]
\surroundwithmdframed[outerlinewidth=0pt,
  innerlinewidth=0pt,
  middlelinewidth=2.5pt,
  middlelinecolor=AAUBlue,
  bottomline=false,topline=false,rightline=false]{example}


% ¤¤ Grafteori mm. ¤¤ %
% \usepackage{tkz-graph}
% \usepackage{calc}
% \usetikzlibrary{decorations.markings,arrows.meta}
% \tikzstyle{vertex}=[circle, draw, inner sep=0pt, minimum size=6pt]
% \newcommand{\vertex}{\node[vertex]}
% \newcounter{Angle}
% \tikzset{every loop/.style={min distance=15mm}}

% ¤¤ Algoritmer ¤¤ %
\usepackage{algorithm}
\usepackage[noend]{algpseudocode}
\makeatletter
\def\BState{\State\hskip-\ALG@thistlm}
\makeatother

% ¤¤ Bibliografi ¤¤ %
%\usepackage[backend=bibtex,
%  bibencoding=utf8
%  ]{biblatex}
%\addbibresource{bib/mybib.bib}
%\DeclareNameAlias{sortname}{last-first}
%\DeclareNameAlias{default}{last-first}

% ¤¤ Python ¤¤ %
% Default fixed font does not support bold face
\DeclareFixedFont{\ttb}{T1}{txtt}{bx}{n}{10} % for bold
\DeclareFixedFont{\ttm}{T1}{txtt}{m}{n}{10}  % for normal

% Custom colors
\usepackage{color}
\definecolor{deepblue}{rgb}{0,0,0.5}
\definecolor{deepred}{rgb}{0.6,0,0}
\definecolor{deepgreen}{rgb}{0,0.5,0}

% Euler i Latex
\newcommand{\me}{\mathrm{e}}
\newcommand{\id}{\text{id}}


\mdfsetup{skipabove=10pt,skipbelow=0pt}

% ¤¤ Figure/Table Numbering ¤¤ %
%\makeatletter
%\renewcommand\@memmain@floats{%
%  \counterwithin{figure}{section}
%  \counterwithin{table}{section}}
%\makeatother

% ¤¤ Referencer og kilder ¤¤ %
%\usepackage{varioref}						% Muliggør bl.a. krydshenvisninger med sidetal (\vref)

%\newcommand*{\eqvref}[1]{\eqref{#1} on page~\pageref{#1}}



%\usepackage[noabbrev]{cleveref}
%\usepackage{natbib}							% Udvidelse med naturvidenskabelige citationsmodeller

% ¤¤ Litteraturlisten ¤¤ %
%\bibpunct[,]{}{}{;}{a}{,}{,}
%Harvard henvisning (bl.a. parantestype og seperatortegn)
%\bibliographystyle{Bibtex/harvard}			% Udseende af litteraturlisten.

% ¤¤ Bibliografi ¤¤ %
\usepackage[backend=biber, style=authoryear,
  bibencoding=utf8]{biblatex}
\setlength\bibitemsep{0.5\baselineskip}
\addbibresource{Bibtex/mybib.bib}
\DeclareNameAlias{sortname}{last-first}
\DeclareNameAlias{default}{last-first}

\usepackage{letltxmacro}\LetLtxMacro{\cite}{\parencite} %Laver \cite kommandoen om til \parencite kommandoen
\renewcommand*{\nameyeardelim}{\addcomma\space} %Tilføjer et komma mellem author og year

\AtEveryCite{%
  \let\parentext=\parentexttrack%
  \let\bibopenparen=\bibopenbracket%
  \let\bibcloseparen=\bibclosebracket}

\makeatother

% ¤¤ Indholdsfortegnelse ¤¤ %
\setsecnumdepth{subsection}		 			% Dybden af nummerede overkrifter (part/chapter/section/subsection)
\maxsecnumdepth{subsection}					% Dokumentklassens graense for nummereringsdybde
\settocdepth{subsection} 					% Dybden af indholdsfortegnelsen

%\nouppercaseheads
%\makepagestyle{analyse}
%\makeoddhead{analyse}{Opgave 230}{}{Gruppe G3-106}
%\makeoddfoot{analyse}{}{\thepage}{}

%\nouppercaseheads
%\makepagestyle{lama}
%\makeoddhead{lama}{LAMA - Miniprojekt}{Gruppe G3-106}{e16g3106@math.aau.dk}
%\makeoddfoot{lama}{}{\thepage}{}

\usepackage{framed}

\newcommand\numberthis{\addtocounter{equation}{1}\tag{\theequation}}

% \usepackage{tikz}
% \usetikzlibrary{matrix}



%Pakke til tabeller
% \usepackage[table,xcdraw]{xcolor}

%\usepackage{appendix}
\usepackage{listings}


%Pæn R kode
\definecolor{ForestGreen}{RGB}{37,137,37}
\definecolor{RoyalBlue}{RGB}{65,105,225}
\definecolor{Purple}{RGB}{102,0,102}

\usepackage{bbm}
%\usepackage{courier}
\usepackage{enumitem}
\setlist[description]{leftmargin=\parindent,labelindent=\parindent}


%  \usepackage[usenames,dvipsnames]{color} 
\lstset{ 
    language=R,                     % the language of the code
    literate=~{$\sim$}2,
    %literate={\$}{{\textcolor{blue}{\$}}}1,
    morekeywords={geom, ggplot},
    deletekeywords={index,legend,title,show,order,scale,predict,real,model,time,kappa,cal,/,q,t,binomial,*,col,C,R,c,\$,beta,hat,\%,dt,path, residuals},
    otherkeywords={},
    basicstyle=\small \ttfamily, % the size of the fonts that are used for the code
    numbers=left,                   % where to put the line-numbers
    numberstyle=\tiny\color{black},  % the style that is used for the line-numbers
    stepnumber=1,                   % the step between two line-numbers. If it is 1, each line
                                   % will be numbered
    numbersep=5pt,                  % how far the line-numbers are from the code
    backgroundcolor=\color{white},  % choose the background color. You must add \usepackage{color}
    showspaces=false,               % show spaces adding particular underscores
    showstringspaces=false,         % underline spaces within strings
    showtabs=false,                 % show tabs within strings adding particular underscores
    frame=false,                   % adds a frame around the code
    rulecolor=\color{black},        % if not set, the frame-color may be changed on line-breaks within not-black text (e.g. commens (green here))
    tabsize=2,                      % sets default tabsize to 2 spaces
    captionpos=b,                   % sets the caption-position to bottom
    breaklines=true,                % sets automatic line breaking
    breakatwhitespace=false,        % sets if automatic breaks should only happen at whitespace
    mathescape = true,
    %keywordstyle=\color{RoyalBlue},      % keyword style
    commentstyle=\color{ForestGreen},   % comment style
    stringstyle=\color{Purple}      % string literal style
} 
\newcommand{\dollar}{\mbox{\textdollar}}



%%%%%%% Opsætning af hyperlink til table of contents  %%%%
\usepackage{eso-pic}
\usepackage{ifthen}
\newboolean{linktoc}
\setboolean{linktoc}{true}  %%% uncomment to show answers properly
%\setboolean{linktoc}{false}  %%% comment to show answers properly

\newcommand\AtPageUpperRight[1]{\AtPageUpperLeft{%
 \put(\LenToUnit{\paperwidth},\LenToUnit{-0.3\paperheight}){#1}%
 }}%
\newcommand\AtPageLowerRight[1]{\AtPageLowerLeft{%
 \put(\LenToUnit{\paperwidth},\LenToUnit{0.3\paperheight}){#1}%
 }}%

\ifthenelse{\boolean{linktoc}}%
{%
\AddToShipoutPictureBG{%
   \AtPageUpperRight{\put(-300,207){\hyperref[toc]{\textcolor{white}{Go to ToC}}}}
   %\AtPageLowerRight{\put(-70,-70){\hyperref[toc]{Go to ToC}}}
    }%
}%
{}%

\usepackage{caption}
\usepackage{booktabs}
\usepackage{wrapfig}
\usepackage{graphicx}
\usepackage{lastpage}




\crefname{theorem}{Theorem}{Theorems}
\crefname{equation}{Equation}{Equations}
\crefname{assum}{Assumption}{Assumptions}
\crefname{mydef}{Definition}{Definitions}
\crefname{figure}{Figure}{Figures}
\crefname{section}{Section}{Sections}
\crefname{chapter}{Chapter}{Chapters}
\crefname{table}{Table}{Tables}
\crefname{example}{Example}{Examples}
\crefname{corollary}{Corollary}{Corollaries}
\crefname{lemma}{Lemma}{Lemmas}
\crefname{prop}{Proposition}{Propositions}
\crefname{appendix}{Appendix}{Appendices}


% Kommentarer og rettelser med \todo{}, insæt figur med \missingfigure{tekst i figuren} 
\usepackage[colorinlistoftodos,prependcaption,textsize=tiny]{todonotes}
\usepackage{regexpatch}
%\tracingxpatches%for debugging
\makeatletter
\xpatchcmd{\@todo}{\setkeys{todonotes}{#1}}{\setkeys{todonotes}{inline,#1}}{}{}
\makeatother

